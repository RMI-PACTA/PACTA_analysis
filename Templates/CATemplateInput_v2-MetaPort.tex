%!TEX TS-program = xelatex
\documentclass[10pt,table,a4]{article}\usepackage[]{graphicx}\usepackage[]{color}

\usepackage{alltt}
\usepackage{graphicx}
\usepackage{gensymb}
\usepackage[top=1cm, bottom=1.5cm, left=1.2cm, right=1.2cm]{geometry}
\usepackage[font=small]{caption}
\usepackage{adjustbox}
\usepackage{fancyhdr}
\usepackage{layout}
%\usepackage{booktabs}
%\usepackage{kpfonts}
\usepackage[explicit]{titlesec}
\usepackage{wrapfig}
\usepackage{tcolorbox}
\usepackage{xcolor}
\usepackage{setspace}
\usepackage{parskip}
\usepackage{tikz}
\usepackage{fontspec}
\usepackage{anyfontsize}
\usepackage{hyperref}
\usepackage{multicol}
\usepackage{datetime}
\usepackage{fixltx2e}
\usepackage[sfdefault]{ClearSans}

% Colours
\definecolor{Yellow1}{RGB}{252, 190, 14}
\definecolor{Yellow2}{RGB}{252, 190, 54}
\definecolor{Yellow3}{RGB}{254, 238, 207}

\definecolor{OffBlack}{RGB}{61,61,60}
\definecolor{LightGray}{RGB}{208,208,208}

\definecolor{ColRed}{RGB}{244,123,115}
\definecolor{ColOrange}{RGB}{253,226,145}
\definecolor{ColYellow}{RGB}{255,255,204}
\definecolor{ColGreen}{RGB}{195,214,155}

\pagestyle{fancy}
\fancyhf{}
\fancyhead[R]{\thepage}
\renewcommand{\headrulewidth}{0pt}

\newcolumntype{P}[1]{>{\centering\arraybackslash}p{#1}}


\newcommand{\nearr}[1]{$\color{#1}\nearrow$}
\newcommand{\searr}[1]{$\color{#1}\searrow$}
\newcommand{\uparr}[1]{$\color{#1}\uparrow$}




\setlength{\parskip}{10pt}

%\pagenumbering{gobble}

\newcommand*{\PageHeadingSingleLine}{%
	\begin{tikzpicture}[remember picture,overlay]
	\node[anchor=north west,minimum width=.375cm,minimum height=1.2cm,fill=Yellow1] (RB) at (-1.2,1.2){\Large };
	\node[text=OffBlack, right of=RB, xshift = 18cm, yshift=0.75cm] at (0,0){\thepage};
	\end{tikzpicture}}
\newcommand*{\PageHeadingDoubleLine}{%
	\begin{tikzpicture}[remember picture, overlay]
	\node[anchor=north west,minimum width=.375cm,minimum height=1.9cm,fill=Yellow1] (RB) at (-1.2,1.2){\Large };
	\node[text=OffBlack, right of=RB, xshift = 18cm, yshift=0.6cm] at (0,0){\thepage};
	\end{tikzpicture}}
\newcommand{\HeaderSingle}[1]{
	\PageHeadingSingleLine 
	
	\vspace{-1.2cm}
	{\Large\textbf{#1}}
	\vspace{.2cm}}
\newcommand{\HeaderDouble}[2]{
	\PageHeadingDoubleLine
	
	\vspace{-1.2cm}
	{\Large\textbf{#1 \\[2pt] #2}}
	\vspace{.45cm}}
\newcommand*{\SectionHeadingBox}[1]{%
	\begin{tikzpicture}[remember picture, overlay]
	\node[anchor=north west,minimum width=.375cm,minimum height=#1,fill=Yellow1] (RB) at (-1.2,-16){\Large };
	\end{tikzpicture}
	\vspace{.8cm}}
\newcommand{\SectionHeading}[2]{
	\SectionHeadingBox{3cm}
	
	\vspace{15.7cm}
	\textbf{{\Huge #1 \\[6pt]\Huge #2}}}
\newcommand{\SectionHeadingDouble}[3]{
	\SectionHeadingBox{4cm}
	
	\vspace{15.7cm}
	\textbf{{\Huge #1 \\[6pt]\Huge #2\\[6pt]\Huge #3}}}
\newcommand{\PageFooterFirst}{% The number indicates the sector..... 
	\begin{tikzpicture}[remember picture, overlay]
	\node[text=OffBlack,above = .8cm, left = 6cm,font=\bf\small,align=center] at (current page.south){01-INTRODUCTION};
	\node[text=LightGray,above = .8cm, left = 1.7cm,font=\bf\small,align=center] at (current page.south){02-CURRENT EXPOSURE};
	\node[text=LightGray,above = .8cm, left = -1cm,font=\bf\small,align=center] at (current page.south){03-5YR TREND};
	\node[text=LightGray,above = .8cm, left = -5cm,font=\bf\small,align=center] at (current page.south){04-EXPOSURE IN 5YRS};
	\node[text=LightGray,above = .8cm, left = -9cm,font=\bf\small,align=center] at (current page.south){05-COMPANY RESULTS};	
	\node[left=6cm,minimum width = 3.0cm, minimum height =0.01cm, fill = Yellow1] at (current page.south){};
	\end{tikzpicture}
}
\newcommand{\PageFooterSecond}{% The number indicates the sector..... 
	\begin{tikzpicture}[remember picture, overlay]
	\node[text=LightGray,above = .8cm, left = 6cm,font=\bf\small,align=center] at (current page.south){01-INTRODUCTION};
	\node[text=OffBlack,above = .8cm, left = 1.7cm,font=\bf\small,align=center] at (current page.south){02-CURRENT EXPOSURE};
	\node[text=LightGray,above = .8cm, left = -1cm,font=\bf\small,align=center] at (current page.south){03-5YR TREND};
	\node[text=LightGray,above = .8cm, left = -5cm,font=\bf\small,align=center] at (current page.south){04-EXPOSURE IN 5YRS};
	\node[text=LightGray,above = .8cm, left = -9cm,font=\bf\small,align=center] at (current page.south){05-COMPANY RESULTS};	
	\node[left=1.75cm,minimum width = 3.6cm, minimum height =0.01cm, fill = Yellow1] at (current page.south){};
	\end{tikzpicture}
}
\newcommand{\PageFooterThird}{% The number indicates the sector..... 
	\begin{tikzpicture}[remember picture, overlay]
	\node[text=LightGray,above = .8cm, left = 6cm,font=\bf\small,align=center] at (current page.south){01-INTRODUCTION};
	\node[text=LightGray,above = .8cm, left = 1.7cm,font=\bf\small,align=center] at (current page.south){02-CURRENT EXPOSURE};
	\node[text=OffBlack,above = .8cm, left = -1cm,font=\bf\small,align=center] at (current page.south){03-5YR TREND};
	\node[text=LightGray,above = .8cm, left = -5cm,font=\bf\small,align=center] at (current page.south){04-EXPOSURE IN 5YRS};
	\node[text=LightGray,above = .8cm, left = -9cm,font=\bf\small,align=center] at (current page.south){05-COMPANY RESULTS};	
	\node[left=-.9cm,minimum width = 2.2cm, minimum height =0.01cm, fill = Yellow1] at (current page.south){};
	\end{tikzpicture}
}
\newcommand{\PageFooterFourth}{% The number indicates the sector..... 
	\begin{tikzpicture}[remember picture, overlay]
	\node[text=LightGray,above = .8cm, left = 6cm,font=\bf\small,align=center] at (current page.south){01-INTRODUCTION};
	\node[text=LightGray,above = .8cm, left = 1.7cm,font=\bf\small,align=center] at (current page.south){02-CURRENT EXPOSURE};
	\node[text=LightGray,above = .8cm, left = -1cm,font=\bf\small,align=center] at (current page.south){03-5YR TREND};
	\node[text=OffBlack,above = .8cm, left = -5cm,font=\bf\small,align=center] at (current page.south){04-EXPOSURE IN 5YRS};
	\node[text=LightGray,above = .8cm, left = -9cm,font=\bf\small,align=center] at (current page.south){05-COMPANY RESULTS};	
	\node[left=-5cm,minimum width = 3.5cm, minimum height =0.01cm, fill = Yellow1] at (current page.south){};
	\end{tikzpicture}
}
\newcommand{\PageFooterFifth}{% The number indicates the sector..... 
	\begin{tikzpicture}[remember picture, overlay]
	\node[text=LightGray,above = .8cm, left = 6cm,font=\bf\small,align=center] at (current page.south){01-INTRODUCTION};
	\node[text=LightGray,above = .8cm, left = 1.7cm,font=\bf\small,align=center] at (current page.south){02-CURRENT EXPOSURE};
	\node[text=LightGray,above = .8cm, left = -1cm,font=\bf\small,align=center] at (current page.south){03-5YR TREND};
	\node[text=LightGray,above = .8cm, left = -5cm,font=\bf\small,align=center] at (current page.south){04-EXPOSURE IN 5YRS};
	\node[text=OffBlack,above = .8cm, left = -9cm,font=\bf\small,align=center] at (current page.south){05-COMPANY RESULTS};	
	\node[left=-8.9cm,minimum width = 3.4cm, minimum height =0.01cm, fill = Yellow1] at (current page.south){};
	\end{tikzpicture}
}

\setmainfont{ClearSans}
\color{OffBlack}


% Box with a shaded border
\newtcolorbox{shadedbox}{colback=Yellow3}

\begin{document}
	\section*{} % Front Page
	
	
	\thispagestyle{empty}
	\pagecolor{Yellow1}
	
	\vspace{-2.6cm}
	
	\begin{tikzpicture}[remember picture, overlay]
	\vspace{7cm}
	\node[below=-.6cm] (CN) at (current page.north){\adjincludegraphics[height=15cm,trim={0cm 0cm 0cm 2cm},clip]{ReportGraphics/FrontPageCA.jpeg}
	};
	
	\node[below = 4.5cm, right,align = left, white, text width=10cm](TN) at (current page.center){
		\Huge{\textbf{2° SCENARIO ANALYSIS}}\\[10pt]
		
		\vspace{2cm}
		{\baselineskip=20pt\Huge\raggedright{\textbf{InsuranceCompanyName}\par}}};
	
	\node[below = 3.8cm,right= .12cm, minimum width = .8cm, minimum height =0.3cm,fill= OffBlack] at (current page.center){};
	
	\end{tikzpicture}
	
	
	\begin{tikzpicture}[remember picture,overlay]
	\node[anchor=south west, yshift = 0cm ,minimum width=21.6cm,minimum height=4cm,fill=white] (RB) at (current page.south west){};
	\node[left= 7cm, above=1.cm] (WS) at (current page.south){\adjincludegraphics[height=1.8cm,trim={0cm 0cm 0cm 0cm},clip]{ReportGraphics/Logo_front}};
	\node[ below = 1.8cm, minimum width = 3.5cm, minimum height =0.05cm, fill = Yellow1] at (WS){};
	\end{tikzpicture}
	
	
	
	
	
	\newpage
	\pagecolor{white}
	\section*{} % Disclaimer p2
	
	\begin{center}
		\textbf{2Dii PACTA MODEL}
		
		\textbf{Important Information \& Legal Disclaimer: MODEL OUTPUT REPORTS}
		
		\textbf{IMPORTANT INFORMATION}
		
	\end{center}
	
	
	The 2Dii PACTA Model generates a limited 'point in time' estimate of the relative alignment of the Revealed Plans of Securities in the Scope versus the economic trends embodied in the 2C Scenario(s), as identified by external data and scenario providers. 
	
	\textbf{EXCLUSION OF LIABILITY: }TO THE EXTENT PERMITTED BY LAW WE WILL NOT BE LIABLE TO ANY USER FOR ANY LOSS OR DAMAGE, WHETHER IN CONTRACT, TORT (INCLUDING NEGLIGENCE), BREACH OF STATUTORY DUTY OR OTHERWISE, EVEN IF FORESEEABLE, ARISING UNDER OR IN CONNECTION WITH USE OF OR RELIANCE ON ANY INFORMATION, DATA OR CONTENT OBTAINED VIA OUR SERVICES, INCLUDING (WITHOUT LIMITATION) THE MODELLING OUTPUTS STATED IN THIS REPORT.
	
	\textbf{No forecast or prediction: }The PACTA Model does not purport to generate, nor does this Report contain or comprise, statements of fact, forecasts or predictions. The PACTA Model provides a 'point in time' analysis of economic and commercial variables that are inherently dynamic and variable over time. 2Dii neither makes nor implies any representation regarding the likelihood, risk or expectation of any future matter. To the extent that any statements made or information contained in this Report might be considered forward-looking in nature, they are subject to risks, variables and uncertainties that could cause actual results to differ materially. You are cautioned not to place undue reliance on any such forward-looking statements, which reflect our assumptions only and those of our data and scenario providers as of the date of modelling.
	
	\textbf{No financial advice: }The information contained in this Report does not comprise, constitute or provide, nor should it be relied upon as, investment or financial advice, credit ratings, an advertisement, an invitation, a confirmation, an offer or a solicitation, or recommendation, to buy or sell any security or other financial, credit or lending product or to engage in any investment activity, or an offer of any financial service. This Report does not purport to quantify risk to the portfolio (or any part thereof), nor make any representation in regards to the performance, strategy, prospects, creditworthiness or risk associated with any investment, nor their suitability for purchase, holding or sale in the context of any particular portfolio. The Modelling Outputs reflected in this Report are provided with the understanding and expectation that each investor will, with due care, conduct its own investigation and evaluation of each security or other instrument that is under consideration for purchase, holding or sale. 
	
	\textbf{Scope Securities: }The PACTA Model is limited in its scope and application. It does not consider all securities across all sectors, nor all securities within those sectors. The PACTA Model applies only to the Scope Securities set out in the Methodology Statement, as updated from time to time. 
	
	\textbf{2C Scenario(s): }The PACTA Model will apply one or more 2C Scenarios, as set out in the Methodology Statement. The choice of any 2C Scenarios should not be taken as any endorsement of those scenarios, nor any statement as to the accuracy or completeness of those scenarios' methodologies or assumptions, nor as a general preference of those scenarios over any other economic scenarios. The analysis provided by the PACTA Model may be carried out using other economic scenarios, and users must form their own view as to the decarbonisation scenarios, trajectories and models that are most appropriate to their portfolio. No explicit or implicit assumption is made in relation to the current or future alignment of the 2C Scenarios with climate-related policies of any government at international, national or sub-national level.
	
	\textbf{TCFD: }Use of the PACTA Modelling Tool may support you in initiatives undertaken with regard to the Recommendations of G20 Financial Stability Board's Taskforce on Climate-related Financial Disclosures (TCFD). However, its use in isolation does not purport to provide ‘TCFD compliance’. 
	
	
	\newpage
	\section*{} % Exec Summary p3
	\HeaderSingle{EXECUTIVE SUMMARY}
	
	\begin{multicols}{2}
		\textbf{This report provides a 2°C scenario analysis of the aggregated investments of insurers operating in California.} 
		
		It responds to the recommendations of the G20 Financial Stability Board Task Force on Climate-related Financial Disclosures (TCFD). The California Department of Insurance engaged the 2° Investing Initiative to undertake a scenario analysis of the investment portfolios of insurers operating in California with over \$100 million in premiums. This report provides results for the aggregated investments of 672 insurers.

		The outputs provided in this report – based on the scope summarized in the table on the right – provide an analysis of the portfolio relative to an economic transition consistent with limiting global warming to 2°C above pre-industrial levels. The analysis answers three questions: 
		
		\begin{enumerate}
			\item{What is the current exposure of the portfolio to economic activities affected by the transition to a 	low-carbon economy? (Section 2)}
			\item{Does the portfolio increase or decrease alignment with a 2°C transition over the next 5 years? 	(Section 3)}
			\item{What is the expected future exposure to high- and low-carbon economic activities? (Section 4)}
		\end{enumerate}
		
		The analysis covers two asset classes: listed equity and 'fixed income', which includes corporate bonds as well as bonds of the largest government and municipal power producers. Notably excluded from this analysis (shown in grey below) are insurance subsidiaries of insurance companies in the equity portfolios, sovereign bonds, and alternative investments. 

		
		\begin{center}
			{\rowcolors{2}{white}{Yellow3}
				\setlength{\tabcolsep}{10pt} % Default value: 6pt
				\renewcommand{\arraystretch}{1.5} % Default value: 1
				\begin{tabular}{ p{.35\linewidth} p{.49\linewidth} }
					\hline
					\multicolumn{2}{c}{\textbf{Scope of Analysis}} \\
					\hline
					Insurer Name & InsuranceCompanyName \\ 
					Size of Portfolio & SizeofPortfolio \\ 
					Scenario & IEA 2° Scenario \\ 
					Geography - \newline Financial Assets & Global \\ 
					Geography - \newline Economic Assets & Global \\ 
					Asset Class & AssetClass \\ 
					Peers & 672 Insurers operating in CA with premiums over \$100 million \\
					Portfolio Timestamp & 12.31.2016 \\ 
					Date of Analysis & TodaysDate \\ 
					\hline
				\end{tabular}
			}
			
		\end{center}
		
		
		
		The graph on the bottom left shows the share of the total equity and fixed income investments that are included in this analysis, referred to as "the portfolio" (light and dark blue together). Within this portfolio, the 2°C scenario analysis focuses on the fossil fuel, power, and automotive sectors (dark blue), which account for roughly 90\% of energy-related CO\textsubscript{2}-emissions in a typical portfolio. The chart on the bottom right shows the share of companies active in each of these sectors in the analyzed portfolio.
		
		
		
		
		
		
		
		\vspace{1.2cm}
		
		
	\end{multicols}	
	
	
	
	\begin{multicols}{2}
		
		\textbf{The figure below shows the share of the total fixed income and equity investments included in the analysis.} AnalysisCoverage\% of investments are included in the analysis ("the portfolio"). %ClimateRelevant\% are in the fossil fuels, power, and automotive sectors.
		
		\textbf{The figure below shows the share of the fossil fuel, power, and automotive sectors in the portfolio.}}
		
	\end{multicols}	
	
	\begin{multicols}{2}
		\centering{\adjincludegraphics[width = .9\linewidth,trim={0cm 0.2cm 0cm .2cm},clip]{CAFigures/Fig01}	}
		
		\centering{\adjincludegraphics[width = .9\linewidth,trim={0cm 0.2cm 0cm .2cm},clip]{CAFigures/Fig02}	}
		
	\end{multicols}	
	
	\newpage
	\section*{} % Exec Summary p4
	\HeaderSingle{EXECUTIVE SUMMARY}
	
	\begin{multicols}{2}
		
		\textbf{The figure below shows the estimated percent of the portfolio exposed to activities across the fossil fuel, power, and automotive sectors.}
		
		The percentages are compared to the market portfolio. The market portfolio results are calculated based on the exposure of the global universe of assets in the listed equity and fixed income markets to the fossil fuel, power, and automotive sectors.
		
		The results show the share of the portfolio potentially exposed to transition risks in each of these sectors. The results are calculated by first calculating the exposure of the portfolio to companies active in the fossil fuel, automotive, and power sector, and then calculating the specific technology exposure on the basis of the breakdown of these companies’ asset base. 
		
		A value higher than the market portfolio suggests the portfolio is more exposed to transition risk than the market average. A value lower than the market portfolio suggests the portfolio is less exposed, all other things being equal. As will be outlined in the next two sections, the extent to which these risks will materialize is likely to be at least in part a function of the evolution of the companies’ activities over time. 
		
		\textit{Note: In the graphs below and throughout this report, 'ICE' refers to Internal Combustion Engine (i.e., gasoline and diesel) vehicles.}
		
		
	\end{multicols}
	
	\textbf{Current exposure of the fixed income portfolio to high-carbon and low-carbon activities, as a \% of the portfolio, compared to the fixed income market}	%CBSpecificS
	
	\includegraphics[trim = {0 0cm 0 0cm},width=1\linewidth]{CAFigures/Fig05} %CBSpecificE
	
	
	\textbf{Current exposure of the equity portfolio to high-carbon and low-carbon activities, as a \% of the portfolio, compared to the listed equity market} %EQSpecificS
	
	\includegraphics[trim = {0 0cm 0 0cm},width=1\linewidth]{CAFigures/Fig04}	%EQSpecificE	
	
	
	
	\newpage
	\section*{} % 1st Section
	\SectionHeading{SECTION 1:}{INTRODUCTION}
	
	
	\newpage
	\section*{} % Report Contents p6
	\HeaderSingle{REPORT CONTENTS}
	
	\begin{multicols}{2}
		
		\textbf{This report provides a 2°C scenario analysis, following the recommendations of the G20's Financial Stability Board Task Force on Climate-related Financial Disclosures (TCFD). Specifically, it seeks to inform the reader about four issues.}
		
		\begin{enumerate}
			\item{\textbf{What is the current exposure of the portfolio to economic activities affected by the transition to a low-carbon economy? (Section 2) }
			}
			
			The first part of the report summarizes the exposures of the portfolio (in terms of \% of the portfolio) to business activities potentially affected by the transition to a low-carbon economy and by extension its exposure to transition risk. Specifically, it will quantify the percent of the portfolio exposed to low-carbon and high-carbon activities across the fossil fuel, power, and automotive sectors. The results will be presented relative to the market portfolio. For fossil fuels, the analysis will also show the distribution of results across all analyzed insurance companies and the aggregated portfolios' position in that distribution. 
			
			\item{\textbf{Does the portfolio increase or decrease its alignment with a 2°C transition over the next 5 years? (Section 3)}
			}
			
			The second part of the report will quantify the extent to which the portfolio is building or reducing risk in terms of being aligned / misaligned with the 2°C scenario pathway over the next 5 years across key business activities. The analysis will focus on technologies in the fossil fuel sector (oil production, gas production), electric power sector (coal power, gas power, nuclear power, renewables power), and automotive sector (internal combustion engine vehicles and electric vehicles). The analysis will compare the currently planned production or investment trend in the portfolio with the production or investment trend that would be required under the 2°C scenario. 
			
			\item{\textbf{What is the expected future exposure to high- and low-carbon economic activities based on the current revealed production and investment plans of the companies in the portfolio? (Section 4)}
			}
			
			Section 4 of this report will quantify the expected evolution of the portfolio's exposure to high-carbon and low-carbon activities in 5 years (2023) based on the current revealed production and investment plans of companies in the fossil fuel, power, and automotive sectors. The section will compare the expected mix of the portfolio’s exposure to high-carbon and low-carbon activities relative to the 2°C scenario.
			
			\item{\textbf{What is driving the results? (Section 5)}}
			
			Section 5 will provide background as to the securities and companies that are driving the results presented in the previous sections, including additional analysis on individual companies’ profiles. 
			
			You will also be able to find further background information on the scenarios and modelling at the end of the report (Section 6).
			
			
		\end{enumerate}
		
		
	\end{multicols}
	
	%	\vspace{1cm}
	\begin{tikzpicture}[remember picture, overlay]
	\node[anchor=north west,minimum width=.375cm,minimum height=6.5cm,fill=Yellow1] (ToC) at (-1.2,-.4){};
	\end{tikzpicture}	
	
	\begin{minipage}[t]{.5\linewidth}
		\textbf{Section 1: }Introduction\\
		
		\textbf{Section 2: }The current exposure\\
		
		\textbf{Section 3: }Trajectory of the portfolio relative to a 2°C scenario\\
		
		\textbf{Section 4: }The expected future exposure in 2023\\
		
		\textbf{Section 5: }Company exposure\\
		
		\textbf{Section 6: }Background to the model\\
	\end{minipage}
	
	\PageFooterFirst
	
	\newpage
	\section*{} %CONTEXT p7
	\HeaderSingle{CONTEXT}
	
	\begin{multicols}{2}
		\textbf{Background.} In June 2017, the G20 Financial Stability Board Task Force on Climate-related Financial Disclosures (TCFD) recommended that financial institutions perform scenario analysis on their portfolios to assess financial risks related to climate change. The TCFD grouped climate-related risks into two categories: physical and transition risks. Transition risks are risks generated by the policy, technology, market, and regulatory changes likely to accompany the transition to a low carbon economy. As part of its supervisory role, the California Department of Insurance engaged the 2° Investing Initiative (2Dii) to undertake a scenario analysis of the investment portfolios of insurance companies operating in California with more than \$100 million in premiums.  As part of this assessment, individual scenario analysis reports will be sent to insurers with the 100 largest investment portfolios, and any insurer operating in California can request their own report (see ’Notes and Disclaimer’ section).
		
		\textbf{Goal.} The goal of the scenario analysis was to assess the insurers’ exposure to transition risk, individually and as a whole, based on estimated current and future exposure to high-carbon and low-carbon activities. This report provides results for the aggregated investments of the 672 insurers included in the analysis.  
		
		\textbf{Approach.} The key elements of the analysis are:
		
		\begin{itemize}
			\item{\textit{Current and planned production and investment trends.} Current and planned production plans (for the fossil fuel and automotive sector) and current power capacity as well as new capacity additions (for the power sector) for the next 5 years were sourced from commercial business intelligence databases. These data providers collect capacity and production forecasts at the physical asset level, including barrels of oil by field, cars by model and factory, and new capacity by power plant. 2Dii maps this data to their immediate owners and parent company to generate a company’s aggregate ‘current production profile’ for each technology. These production plans are then linked to the financial securities (equity and fixed income) issued by the company. The asset-level data used for this analysis was retrieved from data providers during the first half of 2017. See the ‘Important Considerations and Limitations‘ section at the end of the report for notes on interpreting power sector capacity data.}
			
			\item{\textit{Allocating the production of physical assets to financial assets.} Based on the share of total equity or debt held in a portfolio, the model allocates a portion of each corporate issuer’s current production plans for each technology to the portfolio. Aggregated to the portfolio level, this is the ‘portfolio's current production profile’ for a technology. This also defines the insurer’s current ‘exposure’ to each technology.}
			
			\item{\textit{From macro-level scenario to micro-level targets. }To calculate production levels consistent with a climate scenario such as the IEA 2°C scenario, the model uses a ‘fair share’ principle that applies the changes specified by the scenario for a given technology and region equally across all owners of physical assets in that region. It creates a set of alternative production and capacity profiles consistent with the scenario, for each technology and company. These alternative profiles are then aggregated to the portfolio level to create the portfolio’s target profile under the scenario. This profile is used to determine the ‘insurer’s target exposure’ to a technology under the scenario. The ‘target exposure’ does not assume any change in the composition of the portfolio: it models the changes in production and investment plans that are required across the different companies held, in order to match the technology deployment roadmap. This report uses the scenarios of the International Energy Agency, specifically the 450S and the 2D scenario.} 
			
		\end{itemize}
		
		\textbf{Results of the scenario analysis.} The portfolio's 'target profile' under the scenario can be compared to the portfolio’s currently revealed production and investment plans by technology to derive the exposure to transition risk as well as the extent to which the portfolio is projected to increase or decrease alignment with the 2°C scenario over the next 5 years. It is this analysis that forms the basis of the subsequent sections, with Section 6 providing further detail on the methodology.	
		\newline
		
	\end{multicols}
	
	\PageFooterFirst	
	\newpage
	\section*{} %CONTEXT p8
	\HeaderSingle{TRANSITION RISK FOR INVESTORS}
	\begin{multicols}{2}
		\textbf{What are transition risks?} Transition risks can be broadly defined as economic and financial risks associated with the transition to a low-carbon economy. The international community has defined a mandate to limit the man-made contribution to global warming to well below 2°C above pre-industrial levels. According to best available science, achieving this objective requires decarbonizing the economy in the course of this century. This decarbonization is set to have significant implications for ‘high-carbon sectors’, most prominent among which are the fossil fuel, power, and transport sectors, contributing the majority of global anthropogenic GHG emissions. 
		
		As the economy decarbonizes, companies that fail to properly anticipate this transition are set to be exposed to economic risks. Companies well-prepared for this transition in turn are set to capitalize from this economic opportunity. Similarly, economic risks may translate into financial risks in financial markets if these risks are not properly anticipated by financial market actors. 
		
		Crucially, the transition to a low-carbon economy is set to already have dramatic impacts in the short- and medium-term. By 2040, in only 22 years, global coal production is set to decline by 46\%, with a more accelerated decline expected in developed markets. Global coal power capacity in turn is similarly set to decline by 41\%. The production of gasoline and diesel vehicles (internal combustion engine or ICE vehicles) is set to decline by 21\%. This decline in high-carbon activity in turn will be accompanied by the commensurate deployment and growth of new technologies. Renewable power capacity and electric vehicle production in turn is set to nearly quadruple in volume by 2040. 
		
		Scenario analysis can help financial institutions assess and ultimately manage the risks and opportunities associated with the transition. In recognition of these risks, scenario analysis has been applied to date by hundreds of financial institutions as well as financial supervisors. It forms the basis of the recommendations of the FSB TCFD. The TCFD notes that “forward-looking assessments of climate-related issues is important for investors and other stakeholders in understanding how vulnerable individual organizations are to transition and physical risks and how such vulnerabilities are or would be addressed. As a result, the Task Force believes that organizations should use scenario analysis to assess potential business, strategic, and financial implications of climate-related risks and opportunities and disclose those, as appropriate, in their annual financial filings” (TCFD Final Report, p. 33). 
		
		To clarify its scenario analysis recommendation, the Task Force explains, “A key type of transition risk scenario is a so-called 2°C scenario, which lays out a pathway and an emissions trajectory consistent with holding the increase in the global average temperature to 2°C above pre-industrial levels (TCFD Final Report, p. 35).” 
		
		It is this premise that forms the basis of this report, highlighting for the portfolio the current exposure to transition risks in the fossil fuel, power, and automotive sectors, the trends in the portfolio over time in these sectors relative to the 2°C scenario, and the expected future exposure on the basis of these trends. While these sectors do not represent all high-carbon activities and sectors, they account for both the largest share in a typical portfolio and the most significant contribution to climate change currently, as well as benefiting from well-developed scenario pathways.
		
		The report does not provide specific estimates as to the potential loss in value that may be realised in the portfolio should these risks materialize, which is obviously associated with significant uncertainty and myriad modelling assumptions. For any individual security, the potential loss may range from 0 to 100\% and may even be associated with positive returns, depending on the adaptive capacity of the company, the anticipation of the trend by financial markets, and the nature of a potential repricing. It is the proper anticipation of these risks that minimizes the loss that this report seeks to contribute to. 
		
		
	\end{multicols}
	
	
	\begin{center}
		{\setlength{\tabcolsep}{10pt} % Default value: 6pt
			\renewcommand{\arraystretch}{1} % Default value: 1
			\begin{tabular}{ p{.24\linewidth}| P{.24\linewidth}| P{.24\linewidth}}
				\hline
				\textbf{Technology} & \textbf{Total Volume Change by 2023} & \textbf{Total Volume Change by 2040} \\ 
				\hline
				Renewable Power & 69\% \nearr{ColGreen} & 354\% \uparr{ColGreen}\\
				Hydro Power & 13\% \nearr{ColGreen} & 59\% \nearr{ColGreen}\\
				Nuclear Power & 17\% \nearr{ColGreen} & 89\% \nearr{ColGreen}\\
				Gas Power & 8\% \nearr{ColGreen} & 31\% \nearr{ColGreen}\\
				Coal Power & -3\% \searr{ColRed} & -41\% \searr{ColRed}\\				
				\hline
				Oil Production & -2\% \searr{ColRed} & -23\% \searr{ColRed}\\
				Gas Production & 5\% \nearr{ColGreen} & 8\% \nearr{ColGreen}\\
				Coal Production & -11\% \searr{ColRed}& -46\% \searr{ColRed}\\ 
				\hline
				ICE Production & -9\% \searr{ColRed} & -21\% \searr{ColRed}\\
				Hybrid Production & 97\% \uparr{ColGreen} & 440\% \uparr{ColGreen}\\
				Electric Production & 105\% \uparr{ColGreen} & 352\% \uparr{ColGreen}\\
				\hline
			\end{tabular}
		}
	\end{center}
	
	\PageFooterFirst
	\newpage
	\section*{} % 2nd Section
	\SectionHeading{SECTION 2:}{THE CURRENT EXPOSURE}
	
	\newpage
	\section*{} % 2° SCENARIO - CURRENT EXPOSURE 2018 p10
	\HeaderDouble{CURRENT EXPOSURE}{COMPARISION TO MARKET}
	
	\begin{multicols}{2}
		
		
		\textbf{This page provides information on the exposure of the portfolio to high-carbon and low-carbon activities in the fossil fuel, power, and automotive sectors. }
		
		These business activities account for roughly 70-90\% of energy-related CO\textsubscript{2}-emissions in the typical investor portfolio. The figures below show the weight of each technology/fuel in the portfolio by asset class. The graphs below highlight the relative weight of high-carbon and low-carbon technologies within each sector. For context, the results for the relevant listed equity and fixed income market are also included in this analysis.
		
		A sector value that is higher than the market portfolio suggests the portfolio is more exposed to transition risk than the market average. A sector value lower than the market portfolio suggests the portfolio is less exposed, all other things being equal. As will be outlined in the next two sections, the extent to which these risks will materialize is likely to be at least in part a function of the evolution of the companies’ activities over time. 
		
		\textit{The results are calculated by first calculating the exposure of the portfolio to companies active in the fossil fuel, automotive, and power sectors, and then calculating the specific technology exposure on the basis of the breakdown of these companies’ asset base (see Fig. below). }
		
		\vspace{-0.1cm}
		\adjincludegraphics[width = 1\linewidth,trim={0cm 0cm 0cm 0cm},clip]{ReportGraphics/CMExplainer}		
		
	\end{multicols}
	
	\vspace{-1.2cm}
	
	\textbf{Technology breakdown of the sectors analyzed in the fixed income portfolio} %CBSpecificS
	
	\vspace{-0.2cm}
	
	\adjincludegraphics[width = 1\linewidth,trim={0cm 0cm 0cm 0cm},clip]{CAFigures/Fig05}	
	%CBSpecificE
	
	
	\textbf{Technology breakdown of the sectors analyzed in the equity portfolio} %EQSpecificS
	
	\vspace{-0.2cm}
	
	\adjincludegraphics[width = 1\linewidth,trim={0cm 0cm 0cm 0cm},clip]{CAFigures/Fig04}
	
	%EQSpecificE

	\vspace{-1.46cm}
	\PageFooterSecond
	\newpage	
	\section*{} % 2°C SCENARIO CURRENT EXPOSURE – COMPARISION TO PEERS p11
	\HeaderDouble{CURRENT EXPOSURE}{COMPARISION TO PEERS}	
	
	\begin{multicols}{2}
		\textbf{This page compares the fossil fuel exposure of the aggregated portfolios to the fossil fuel exposure of individual insurers operating in California.}
		
		It takes the information from the previous page and contextualizes it relative to the other insurance companies covered under this assessment. More specifically, the graphs below isolate the exposure to upstream fossil fuels (coal, oil, and gas production) in the fixed income and equity portfolios. For each asset class, the distribution highlights the range of fossil fuel exposure across insurance companies and relative to the value for the aggregated portfolio. 
		
		
	\end{multicols}
	
	
	\textbf{Distribution of exposure to fossil fuels within all fixed income portfolios} %CBSpecificS
	
	\adjincludegraphics[width = 1\linewidth,trim={0cm 0cm 0cm 0cm},clip]{CAFigures/Fig12}	%CBSpecificE
	
	\vspace{-.8cm}
	\textbf{Distribution of exposure to fossil fuels within all equity portfolios} %EQSpecificS
	
	\adjincludegraphics[width = 1\linewidth,trim={0cm 0cm 0cm 0cm},clip]{CAFigures/Fig11} %EQSpecificE	 
	
	
	\PageFooterSecond
	\newpage
	%	\section*{} % ENVIRONMENTAL RISK CURRENT EXPOSURE –DEBT %CBSpecificS
	%	\HeaderSingle{ENVIRONMENTAL RISK CURRENT EXPOSURE – FIXED INCOME}	
	%
	%		\begin{multicols}{2}
	%			\textbf{Beyond the fossil fuel, power, and automotive sectors, investment portfolios may also have transition and environmental risk exposure across a broader range of sectors.} 
	%			
	%			The following shows the potential environmental risk exposure based on a broader analysis of the sectoral exposure of the fixed income and equity portfolio and those of other California insurance companies. The specific sectors included in this analysis are shown below. 
	%			
	%			\textbf{The following four risk levels are represented in the classification: }
	%		\end{multicols}
	%
	%		
	%		\begin{center}
	%			{\setlength{\tabcolsep}{10pt} % Default value: 6pt
	%				\renewcommand{\arraystretch}{1.5} % Default value: 1
	%				\begin{tabular}{ p{.2\linewidth}| p{.7\linewidth} }
	%					\hline
	%					\textbf{Risk Level} & \textbf{Sector} \\ 
	%					\hline
	%					\cellcolor{ColRed} Immediate Elevated & Independent Power Producers, Coal and Consumable Fuels \\ 
	%					\hline
	%					\cellcolor{ColOrange} Emerging Elevated & Steel, Aluminum, Oil and Gas E\&P, Construction Materials, Diversified Metals and Mining, Auto Manufacturers \\ 
	%					\hline
	%					\cellcolor{ColYellow} Emerging Moderate & Regulated Utilities, Airlines, Integrated Oil and Gas, Paper, Oil and Gas services, Auto Parts, Gas Utilities \\ 
	%					\hline
	%					\cellcolor{ColGreen} Low & Marine, Diversified Chemicals, Industrial Gases, Marine Ports \\ 
	%					\hline
	%				\end{tabular}
	%			}
	%		\end{center}
	%		
	%		\begin{multicols}{2}
	%			\textbf{The following chart presents the percentage of the portfolio by AUM that is rated as Immediate Elevated, Emerging Elevated or Emerging Moderate.} 
	%			
	%			While there is some variation, almost all Californian investors, have some exposure to sectors of elevated risk. While this is primarily only in emerging elevated sectors, it gives an indication that the invested fixed income market is quite exposed to risk. 
	%			
	%			The portfolio is above the average in terms of environmental risk exposure, the average being 11\%. Crucially, the assessment provided here does not build on the 	type of granularity described earlier in terms of high-carbon and low-carbon exposures but remains at sector level. It is thus by design more imprecise than alternative approaches.
	%		\end{multicols}	
	%		
	%		
	%		\textbf{The environmental risk exposure of the portfolio relative to its peers}
	%		
	%		\begin{centering}
	%					
	%			\adjincludegraphics[width = 1\linewidth,trim={0cm 0cm 0cm 0cm},clip]{CAFigures/Fig13}
	%		\end{centering}
	%		
	%\PageFooterSecond
	%\newpage	%CBSpecificE
	\section*{} % 3rd Section 
	\SectionHeadingDouble{SECTION 3:}{TRAJECTORY OF THE PORTFOLIO}{RELATIVE TO A 2°C SCENARIO}
	\newpage
	
	\section*{} % TRAJECTORY – DEBT – POWER p13
	\HeaderSingle{5 YEAR TREND - POWER SECTOR}	
	
	
	\begin{multicols}{2}
		
		The analysis for the portfolio builds on the forward-looking projections of capacity additions by fuel over the next 5 years, as sourced from business intelligence data provider GlobalData. The five year time horizon is a function of the typical investment planning horizon of power capacity additions, recognizing that planning horizons for specific investments may be both longer and shorter. More long-term analysis would thus fail to identify significant further additions currently in the planning pipeline of companies. Excluded from the analysis presented here are planned power capacity additions by companies outside of the power sector (e.g. IT companies building wind parks to power their data centers). The evolution of the portfolio is based on the planned capacity additions by the companies behind the securities in the portfolio, weighted by their relative weight in the portfolio. 
		
		It is important to note that data on ‘announced’ or otherwise officially planned retirements of power assets is not considered in the analysis presented here. This is intentional, given both a dearth of related data, as well as the desire to show the required retirements. For technologies projected to decline under the 2° scenario, the gap between current capacity projections and capacity consistent with the 2° scenario should be seen as an estimate of the capacity that would need to be retired to be in alignment with the 2° scenario. 
		
		As outlined above, the scenarios are based on the global trends, scaled to the portfolio based on the ‘fair share’ approach, where the trend in the macro scenario is translated into a micro target based on the market share of the portfolio. For the power sector, this approach may of course fail to capture changes in market share across asset classes and actors, notably with the rise of household renewable power capacity (e.g. rooftop solar), set to change the power market. While this trend implies that in practice companies are likely to lose market share, this trend is intentionally not internalized in the analysis, in order to document the potential loss of market share under a 2°C scenario – and by extension the potential accumulating transition risk.
		
		Further information on the data and the scenarios is provided in Section 6. 
		
		In a 2°C scenario, the power sector will decarbonize over the long-term in a shift from fossil fuel-based to renewable energy production. The International Energy Agency (IEA) says that in a 2°C scenario:
		
		"Electricity supply worldwide is set to diversify and decarbonise, with low-carbon generation overtaking coal before 2020. Coal-fired power’s share of generation is projected to fall from above 40\% now to 28\% in 2040. By then, wind, solar and bioenergy-based renewables combined increase their market share from 6\% to 20\%" (IEA World Energy Outlook 2016, p. 241).
		
		The mix of technologies will vary greatly based on the scenario. Coal-based power generation will increase under current trends but decreases in a 2°C scenario. Wind and solar would grow more rapidly in a 2°C Scenario.
		
		Equity and fixed income investors are exposed to these trends through the financial instruments issued by power companies. An estimated 28\% of power generation assets are owned by publicly traded companies and 19\% of assets are owned by listed state entities, for example municipal bond issuers (see figure below).
	\end{multicols}
	
	\begin{multicols}{2}
		
		\vspace{-.2cm}
		\textbf{Power generation mix under IEA business as usual and 2DS scenarios for selected technologies}
		
		
		
		
		\adjincludegraphics[width = .95\linewidth,trim={0cm 0cm 0cm .5cm},clip]{ReportGraphics/PowerLine.png}
		
		\textit{\small Source: IEA World Energy Outlook 2016
		}
		
		\textbf{Ownership of global power generation assets}
		\newline
		
		\adjincludegraphics[width = .8\linewidth,trim={0cm 0cm 0cm 0.5cm},clip]{ReportGraphics/PowerPie.png}
		
		\textit{\small Source: IEA analysis and 2Dii, based on Platts, Bloomberg Professional service, Bloomberg New Energy Finance and national sources
		}
		
	\end{multicols}
	
	\PageFooterThird	
	\newpage
	\section*{} % TRAJECTORY – DEBT – POWER   %CBSpecificS p14
	\HeaderDouble{5 YEAR TREND - FIXED INCOME}{POWER}		
	
	\begin{multicols}{2}
		\textbf{The alignment graphs below show the alignment of selected power technologies in the fixed income portfolio relative to the IEA scenarios for 2°C, 4°C and 6°C temperature change and the global fixed income market.} 
		
	\end{multicols}	
	
	
	
	\begin{minipage}[t]{.49\linewidth}
		\textbf{Trajectory of Coal Power Capacity }
		
		\includegraphics[trim = {0 0cm 0 0},width=1\linewidth]{CAFigures/Fig14}
		
		\textbf{Trajectory of Renewable Power Capacity }
		
		\includegraphics[trim = {0 0cm 0 0},width=.99\linewidth]{CAFigures/Fig15}
	\end{minipage}	
	\hspace{.02\linewidth}
	\begin{minipage}[t]{.49\textwidth}
		\textbf{Trajectory of Gas Power Capacity }
		
		\includegraphics[trim = {0 0cm 0 0},width=1\linewidth]{CAFigures/Fig16}
		
		\textbf{Trajectory of Nuclear Power Capacity }
		
		\includegraphics[trim = {0 0cm 0 0},width=1\linewidth]{CAFigures/Fig17}
		
	\end{minipage}
	
	\vspace{-0.4cm}
	\begin{center}
		\includegraphics[trim = {0 0cm 0 0},width=.65\linewidth]{ReportGraphics/246Legend.png}
	\end{center}
	
	
	\PageFooterThird
	\newpage %CBSpecificE
	\section*{} % TRAJECTORY – EQUITY – POWER %EQSpecificS p15
	\HeaderDouble{5 YEAR TREND - EQUITY}{POWER}		
	
	\begin{multicols}{2}
		\textbf{The alignment graphs below show the alignment of selected power technologies in the equity portfolio relative to the IEA scenarios for 2°C, 4°C and 6°C temperature change and the global listed equity market.} 
		
	\end{multicols}
	
	
	
	\begin{minipage}[t]{.49\linewidth}
		\textbf{Trajectory of Coal Power Capacity }
		
		\includegraphics[trim = {0 0cm 0 0},width=1\linewidth]{CAFigures/Fig22}
		
		\textbf{Trajectory of Renewable Power Capacity }
		
		\includegraphics[trim = {0 0cm 0 0},width=.99\linewidth]{CAFigures/Fig23}
	\end{minipage}	
	\hspace{.02\linewidth}
	\begin{minipage}[t]{.49\textwidth}
		\textbf{Trajectory of Gas Power Capacity }
		
		\includegraphics[trim = {0 0cm 0 0},width=1\linewidth]{CAFigures/Fig24}
		
		\textbf{Trajectory of Nuclear Power Capacity }
		
		\includegraphics[trim = {0 0cm 0 0},width=1\linewidth]{CAFigures/Fig25}
		
	\end{minipage}
	
	\vspace{-0.6cm}
	\begin{center}
		\includegraphics[trim = {0 0cm 0 0},width=.65\linewidth]{ReportGraphics/246Legend.png}
	\end{center}
	
	
	\PageFooterThird
	\newpage %EQSpecificE
	\section*{} % TRAJECTORY – FI – FOSSIL FUELS AND AUTOMOTIVE     %CBSpecificS p16
	\HeaderDouble{5 YEAR TREND - FIXED INCOME}{FOSSIL FUELS AND AUTOMOTIVE}	
	
	\begin{multicols}{2}
		\textbf{The alignment graphs below show the alignment of selected fossil fuels and automobile technologies in the fixed income portfolio relative to the IEA scenarios for 2°C, 4°C and 6°C temperature change and the global fixed income market. } 
		
	\end{multicols}		
	
	\begin{center}
		\textbf{Fossil Fuel Sector}
	\end{center}
	
	\begin{minipage}[t]{.49\linewidth}
		\textbf{Trajectory of Oil Production }
		
		\includegraphics[trim = {0 0cm 0 0},width=1\linewidth]{CAFigures/Fig18}
		
	\end{minipage}	
	\hspace{.02\linewidth}
	\begin{minipage}[t]{.49\textwidth}
		\textbf{Trajectory of Gas Production }
		
		\includegraphics[trim = {0 0cm 0 0},width=1\linewidth]{CAFigures/Fig19}
		
	\end{minipage}
	
	
	\begin{center}
		\textbf{Automotive Sector}
	\end{center}
	
	\begin{minipage}[t]{.49\linewidth}
		\textbf{Trajectory of Combustion Engine Vehicle Production}
		
		\includegraphics[trim = {0 0cm 0 0},width=1\linewidth]{CAFigures/Fig20}
		
	\end{minipage}	
	\hspace{.02\linewidth}
	\begin{minipage}[t]{.49\textwidth}
		\textbf{Trajectory of Electric Vehicle Production}
		
		\includegraphics[trim = {0 0cm 0 0},width=1\linewidth]{CAFigures/Fig21}
		
	\end{minipage}		
	
	\vspace{-.6cm}
	\begin{center}
		\includegraphics[trim = {0 0cm 0 0},width=.65\linewidth]{ReportGraphics/246Legend.png}
	\end{center}
	
	\PageFooterThird
	\newpage %CBSpecificE
	
	\section*{} % TRAJECTORY – EQUITY – FOSSIL FUELS AND AUTOMOTIVE %EQSpecificS p17
	\HeaderDouble{5 YEAR TREND - EQUITY}{FOSSIL FUELS AND AUTOMOTIVE}	
	
	\begin{multicols}{2}
		\textbf{The alignment graphs below show the alignment of selected fossil fuels and automobile technologies in the equity portfolio relative to the IEA scenarios for 2°C, 4°C and 6°C temperature change and the global listed equity market. } 
		
	\end{multicols}		
	
	\begin{center}
		\textbf{Fossil Fuel Sector}
	\end{center}
	
	\begin{minipage}[t]{.49\linewidth}
		\textbf{Trajectory of Oil Production }
		
		\includegraphics[trim = {0 0cm 0 0},width=1\linewidth]{CAFigures/Fig26}
		
	\end{minipage}	
	\hspace{.02\linewidth}
	\begin{minipage}[t]{.49\textwidth}
		\textbf{Trajectory of Gas Production }
		
		\includegraphics[trim = {0 0cm 0 0},width=1\linewidth]{CAFigures/Fig27}
		
	\end{minipage}
	
	
	\begin{center}
		\textbf{Automotive Sector}
	\end{center}
	
	\begin{minipage}[t]{.49\linewidth}
		\textbf{Trajectory of Combustion Engine Vehicle Production}
		
		\includegraphics[trim = {0 0cm 0 0},width=1\linewidth]{CAFigures/Fig28}
		
	\end{minipage}	
	\hspace{.02\linewidth}
	\begin{minipage}[t]{.49\textwidth}
		\textbf{Trajectory of Electric Vehicle Production}
		
		\includegraphics[trim = {0 0cm 0 0},width=1\linewidth]{CAFigures/Fig29}
		
	\end{minipage}		
	
	\vspace{-.6cm}
	\begin{center}
		\includegraphics[trim = {0 0cm 0 0},width=.65\linewidth]{ReportGraphics/246Legend.png}
	\end{center}
	
	\PageFooterThird
	\newpage %EQSpecificE
	\section*{} % 4th SECTION 
	\SectionHeadingDouble{SECTION 4:}{THE EXPOSURE OF THE PORTFOLIO}{TO 2°C SCENARIOS IN 2023}
	
	
	\newpage
	\section*{} % FUTURE TECHNOLOGY SHARE p19
	\HeaderSingle{FUTURE TECHNOLOGY SHARE}
	
	\begin{multicols}{2}
		\textbf{The figure below shows the estimated exposure in 2023 to high-carbon and low-carbon technologies for the fossil fuel, power, and automotive sectors, in both the fixed income and equity portfolios. }
		
		The results are a function both of the starting point of the exposure (Section 2) and the evolution of the exposure over time (Section 3) based on current revealed investment and production plans for all technologies. For this report, the "Portfolio" and "All Insurers" columns show the relative exposure of the aggregated fixed income and equity portfolios across technologies / fuels. For reports sent to individual insurers, the "Portfolio" column will provide results specific to the insurer's portfolio, which can then be compared to both the aggregated portfolio ("All Insurers") as well as the expected market fuel mix under a 2°C transition in 2023. 
		
		As highlighted previously, the analysis does not include assumptions around changes in portfolio composition. Rather, it is limited to how the portfolio's  exposure to high-carbon and low-carbon technologies is set to change over time as a function of changes in companies’ exposures, independent of portfolio composition changes. The results help contextualize the share of the sectoral exposure in 2023 exposed to transition risks in terms of the share of activities that can be classified as either high-carbon or low-carbon. Given the marginal nature of renewable activities across oil and gas companies, this share has not been considered in the analysis, although it may over time represent a growing share. 
		
		
	\end{multicols}
	
	%CBSpecificS
	\textbf{Fixed Income}
	
	
	\begin{center}
		\includegraphics[trim = {0 0cm 0 0},width=1\linewidth]{CAFigures/Fig10}
	\end{center}
	%CBSpecificE
	
	%EQSpecificS
	\textbf{Equity}
	
	
	\begin{center}
		\includegraphics[trim = {0 0cm 0 0},width=1\linewidth]{CAFigures/Fig09}
	\end{center}
	%EQSpecificE
	\PageFooterFourth
	\newpage 
	\section*{} % 5th COMPANY RESULTS
		\SectionHeading{SECTION 5:}{COMPANY EXPOSURE}
		
		\newpage
		\section*{}
		\HeaderSingle{COMPANY EXPOSURE}
		
		The results presented in the previous section are a function of the specific portfolio exposures to securities and the companies that issue them across all analyzed portfolios. The results thus present an aggregated picture of the weighted trajectory of each individual company in the aggregated portfolios. These companies of course will each have unique exposures, both in terms of their current asset base and the trajectory of their production or production capacity profile. The capital allocation choice made by insurance companies operating in California thus impacts these results not just in their sectoral exposures to different economic activities but also in the individual securities within each sector. 
		In order to help each individual insurance company understand their results, as part of the reports sent to them this section will provide portfolio-specific information on the companies these insurance companies are exposed to as well as their exposure to high-carbon and low-carbon assets.
		
		In the equity portfolios for example, the insurance companies analyzed have exposures to companies that account for around one-third of global oil and gas production, and over 90\% of global auto production. 
		
		Assessing that universe requires a granular analysis of the link between individual assets, the companies that own them and their parent companies, as well as the financial instruments associated with these assets. The figure below highlights the interwoven links within the analysis.
		
	
		\includegraphics[trim = {0 0cm 0 0},width=1\linewidth]{ReportGraphics/Sec5Backup.png}
	
		\PageFooterFifth
		\newpage
	
	%AutoSector_ALLE     CompanyChartsE
	\section*{} % 6th BACKGROUND
	\SectionHeading{SECTION 6:}{BACKGROUND TO THE MODEL}
	
	
	\newpage
	\section*{} %BACKGROUND TO THE MODEL
	\HeaderSingle{BACKGROUND TO THE MODEL}
	
	\begin{multicols}{2}
		\textbf{The objective of the assessment framework applied in this scenario analysis is to measure the alignment of financial portfolios with 2°C decarbonisation pathways. The model consists of 3 key elements that are detailed in the following pages.}
		
		
		\begin{itemize}
			
			\item {Scenarios, notably 2°C scenarios, that form the basis of the analysis and define the benchmark against which portfolio 	trends are compared. While in theory a range of scenarios can be applied within the model, in the interest of simplification this analysis relies on the scenarios of the International Energy Agency. These provide targets for each technology at a regional level. }
			
			\item{Financial portfolios and associated financial data to allow for the portfolio assessment. Within this report, the analysis 	is limited to fixed income and equity portfolios. Funds within the portfolios have been identified and the underlying financial data extracted from Morningstar and included as part of the portfolio. }
			
			\item{Physical / industry ‘asset level data’ (current and forward looking) is mapped to companies, parents, and securities. This 	allows the link between financial portfolios and industry and production data (oil and gas production, automotive production, power capacity) to be established. Consequently, this allows a comparison to the 2°C scenarios and a corresponding evaluation of the alignment of the portfolio. }
			
		\end{itemize}
	\end{multicols}
	
	
	\includegraphics[trim = {0 0cm 0 0},width=1\linewidth]{ReportGraphics/SummaryChart}
	
	\begin{multicols}{2}
		\textbf{Allocation Rules.}
		Based on the financial data, the asset level data is allocated to the portfolio to quantify a representative value of what the portfolio physically owns. The assets are allocated to the portfolio based on the weight of the securities in the portfolio. 
		
		\textbf{Benchmarking.}
		Using the allocated production or capacity of technologies within the portfolio in as a starting point, an allowance based on the 	regional scenarios is calculated. This is extrapolated over the next 5 years to create the trajectory and this is compared to the current and future ownership. The variation of the ownership from this benchmark is used as the alignment indicator in the preceding results. 
	\end{multicols}
	
	
	
	\newpage	
	\section*{} %BACKGROUND TO THE MODEL
	\HeaderSingle{BACKGROUND TO THE MODEL}
	
	\begin{multicols}{2}
		
		
		\textbf{Assessing Alignment with a 2°C Transition Pathway. }This analysis assesses the level of alignment with a 2°C transition 	pathway, using two references:
		
		\begin{itemize}
			\item{\textit{The portfolio's ‘own’ 2°C target.} This is the portfolio‘s target production profile ‘under the 2° scenario’: the 	changes required in the production profile of the companies held in the portfolio, in order to meet the target, based on the above-described methodology. While the 2°C scenario is the focus of this analysis, the target profiles for a 4°C and a 6°C scenario are also calculated to provide further context. Since the securities held and their weight in the portfolio are identical for the portfolio and its alternative versions, comparing them shows how aligned or misaligned the current production profiles of companies held in the portfolio are with each scenario.}
			
			\item{\textit{The 2°C benchmark. }This is the target production profile of a ‘market benchmark’ under the 2° scenario. The same 	principle as described above is applied to a ‘benchmark portfolio’: the listed equity market as a whole, or the corporate fixed income market as a whole. Since the securities and their weight in the market portfolio differ from those in the portfolio, this comparison highlights ‘idiosyncratic’ alignment or misalignment. In other words, it shows how the current composition of the portfolio affects the alignment with the different scenarios, when the first reference only stresses the changes requested from the companies.}
		\end{itemize}
		
		
		The alignment or misalignment of a portfolio’s production and exposure to each technology relative to a scenario is one way to better understand insurers’ exposure to energy transition risk. If policy, technology, market, or regulatory changes occur to bring the global real economy in line with the 2°C scenario, misalignment in a given technology would likely change the financial returns associated with those underlying physical assets. However, this analysis only assesses one dimension of energy transition risks: the assets at risk in the real economy. It does not take into account the financial resilience of the company to those changes and its capacity to adapt, which would require further financial analysis.
		
		\textbf{Scenarios.} The IEA’s 450 Scenario (450S) is the most well know climate scenario globally. It defines how climate-relevant technologies - 	essentially energy technologies - must be deployed by 2050 to reach a 50\% probability of limiting warming to 2°C or 3.6°F. In addition to the 450S, the IEA also defines the New Policies Scenario (NPS) and Current Policies Scenario (CPS): other technology roadmaps that correspond to a 50\% probability of maximum 4°C and 6°C warming, respectively. The 450S (also referred as “2° scenario”), NPS (“4° scenario”), and CPS (“6° scenario”) all provide forward-looking projections with enough regional detail to perform scenario analysis for 11 technologies in 3 sectors. The analysis is based on the IEA scenarios for the California Department of Insurance and covers fossil fuel extraction (oil, gas, and coal mining); production of electricity (from coal, gas, petrol, hydro, nuclear, and renewables); and, the production of cars (internal combustion engines - gasoline and diesel, hybrid, and electric).
		
		The IEA historically has assumed significant amounts of nuclear power and carbon capture and storage in their scenarios. While the IEA has updated the names and models in 2017, given that this report uses 2016 portfolio data, 2016 scenarios were applied for this analysis. In addition, the international community has accelerated their global target from the 2°C goal to well below 2°C with a target of 1.5°C. It is important to highlight that each investor can and may want to take an individual view on the likely decarbonization scenario that may or may not relate to the scenarios modelled by the International Energy Agency or others.
		
		The model uses the following indicators from the International Energy Agency scenario against which the portfolio is compared:
		\begin{itemize}
			\item{Electric capacity by fuel expressed in MW (e.g. renewables, coal, gas, oil, hydropower, nuclear);}
			\item{Oil production expressed in barrels of oil produced / year;}
			\item{Gas production expressed in bcf / year;}
			\item{Coal produced expressed in mtoe / year;}
			\item{GHG emissions pathways in a sample of additional sectors (e.g. aviation, shipping, cement, steel).}
		\end{itemize}
		
		
		\textbf{Asset Level Data.} The Asset Level data is sourced from the following data providers: 
		\begin{itemize}
			\item{GlobalData (Power plant data, including plants classified as active, announced, financed, partially active, permitting, temporarily shutdown, under construction, under rehabilitation and modernization, and Oil and Gas production data and forecasts until 2018-2023, as well as coal mining data); }
			\item{WardsAuto (light passenger duty vehicles, including BAU production forecasts 2018-2023); }
			\item{Bloomberg (financial data);}
			\item{S\&P Cross-Reference Services (database matching securities to parents);}
			\item{Morningstar (database on funds). }
			
		\end{itemize}
		
		
		
		
	\end{multicols}
	
	
	
	\newpage
	\section*{} %BACKGROUND TO THE MODEL
	\HeaderDouble{IMPORTANT CONSIDERATIONS AND LIMITATIONS}{WHEN INTERPRETING THESE RESULTS }
	
	\begin{multicols}{2}
		\begin{itemize}
			\item{\textit{Stringency of scenarios.} The use of a given scenario (2°C, 4°C, and 6°C) does not constitute an assumption that this scenario is more likely to prevail than others. Similarly, the choice of IEA scenarios should not be interpreted as an endorsement of the underlying assumptions by 2Dii or the California Department of Insurance. The IEA historically has assumed significant amounts of nuclear power and carbon capture and storage in their scenarios, an assumption that is debated within the energy-climate scientific community. In addition, the international community has accelerated their global target from the 2°C goal to “well below 2°C and towards 1.5°C“. It is important to highlight that each insurer can and may want to take an individual view on the likely decarbonization scenario that may or may not relate to the scenarios modelled by the International Energy Agency.}
			
			\item{\textit{A snapshot rather than forecasts.} The forward-looking production data is based on current ‘revealed’ plans from companies, and is subject to change. The estimates should thus not be interpreted as forecasts, but rather as the current plans of companies as estimated from various sources of information by industry-specific business intelligence experts - who might not know everything about the CEOs’ actual plans. Given the 5 year time horizon, it is likely that these plans will change in some way over time. Similarly, insurers are highly likely to alter the composition of their portfolio over time. Fixed income maturity is usually around 3-7 years. The average holding period of a stock by a fund manager is 20 months on average. However, this analysis seeks to be a point in time assessment of future exposures under current conditions.}
			
			\item{\textit{Power sector projections.} This is a measure of "locked-in" capacity, not a capacity forecast. Distinct from the production data for the fossil fuel and automotive sectors, capacity data for the power sector does not include information on planned retirements. It should therefore be interpreted as a measure of currently "locked-in" capacity and not as a forecast of future capacity. Retirements are not included for several reasons: First, the availability of planned retirement data is highly variable across jurisdictions and regions, to the extent that including no retirement information was deemed more representative of industry capacity than including partial data. Second, in contrast to the fossil fuel sector where oil wells, gas fields, and coal mines cease production when their resource runs out, it is possible for power plants to be announced as retired or even be retired and then resume production. Given the higher level of uncertainty around planned retirements, they are not included in the power sector projections used for this analysis, and capacity projections should thus be interpreted as the potential maximum “lock-in“ from current infrastructure. For technologies projected to decline under the 2° scenario, the gap between current capacity projections and capacity consistent with the 2° scenario should be seen as an estimate of the capacity that would need to be retired to be in alignment with the 2° scenario.} 
			
			\item{\textit{Changes in plans. }The forward-looking data is based on current ‘revealed’ plans from companies and is subject to change. The estimates should thus not be interpreted as final forecasts, but rather the current plans of companies if they don’t change. Another way to interpret the results is the call for action with regard to the required change to align with the 2°C economic trend. Given the 5 year time horizon, there is a high degree of certainty that plans will still change in some way over time. Similarly, the participating financial institutions can of course alter their portfolio exposures over time. The analysis however seeks to be a point in time assessment of future exposures under current conditions.}
			\item{\textit{Ability to capture SRI strategies. }The model takes a diversified ‘market portfolio’ as a basis, focusing on key technologies reflected in the IEA roadmaps. By extension, thematic portfolios invested in breakthrough technologies and / or SRI portfolios with a range of environmental, social, and governmental considerations may not value these elements.}
		\end{itemize}
		
		
	\end{multicols}
	\newpage		
	
	%	\section*{} %INTERPRETATION AND IMPLICATIONS FOR RISK
	%	\HeaderSingle{INTERPRETATION AND IMPLICATIONS FOR RISK}
	%		
	%		\begin{multicols}{2}
	%				Important in the implementation of different actions based on the 2°C scenario analysis is an understanding of the implications for risk and return of the portfolio. It is important to emphasize here that the results presented in this report are explicitly not a risk analysis. In general, the following findings can be summarized as the interaction between risk, return, and the 2°C scenario analysis. 
	%			
	%			\textbf{What is the risk of inaction?}
	%			Although the analysis focuses on alignment with the Paris Agreement in a way that contributes to the general interest, the issue can also be addressed in terms of the financial risk to the investor if the energy transition is not properly anticipated. 
	%			
	%			For investors, the main risk seems to be more pronounced if the 2°C target is not reached. Aviva and the Economist Intelligence Unit analyzed the net impairment loss for financial assets under management at approximately USD60 trillion in a 2°C scenario (Aviva 2015). The TCFD (Task Force on Climate-related Financial Disclosures) initiated by the Financial Stability Committee (FSB) calls these risks ”physical risks”. If the 2°C objective is achieved, these physical risks can be reduced considerably. The cost would be limited to less than USD10 trillion if we remain below 3°C according to the same Aviva / ECIU estimates. 
	%			
	%			However, investment portfolios can then be exposed to what the TCFD calls ”transition risks” - the economic and financial risks associated with the transition to a low-carbon economy. These risks are likely to be particularly pronounced for the most CO\textsubscript{2}-emitting sectors, and thus their investors. Most of these sectors are covered by our analysis in the previous sections. 
	%			
	%			Although the 2°C scenario presented in this report is not directly a financial risk assessment, it can help to better understand the exposure to transition risk faced by investors. It makes it possible to understand whether the necessary transition will be gradual (when the production and investment plans are aligned with the 2°C scenario) or is likely to be abrupt (sudden correction linked to the introduction of new technologies or constraints legal proceedings leading to bankruptcies of established companies). All investment strategies are exposed to potential risks. The scenario analysis reveals how each strategy evaluated is an explicit or implicit bet on a 2°C, 4°C or 6°C scenario. Depending on the trajectory that will ultimately prevail, the portfolios will underperform or outperform. From the point of view of the optimization of the risk/return ratio in the long term, it is essential to be aware of the bet made.
	%			
	%			From a transition risk perspective, the following three questions are important:
	%			
	%			\begin{enumerate}
	%				\item{Is my portfolio over-exposed to transition risks by deviating from the 2°C benchmark?}
	%				\item{If this is the case, which securities in my portfolio are exposed to these risks?}
	%				\item{Should these risks arise, what are possible losses?}
	%			\end{enumerate}
	%			
	%			
	%			The answer to the first question is provided by the analysis presented in the previous pages. There are different approaches to quantifying exposure:
	%			
	%			\begin{itemize}
	%				\item{Based on the method presented in this report, it is possible to isolate the most misaligned sectors and securities with respect to a 2°C trajectory.}
	%				
	%				\item{The rating agency Moody’s developed in 2016 a methodology to classify the different sectors of their fixed income universe according to the risk of downgrade due to environmental risk.}
	%			\end{itemize}
	%			
	%			
	%			\textbf{Asset Pricing and Risk} 
	%			A final question to be considered is: what is the potential value at risk within the climate relevant sectors if a 2° C scenario materializes? This requires additional financial analysis. In particular, assumptions must be made as to how the market has already (or not) integrated these risks into the current price of financial assets. There are several research papers on the subject, published by financial analysts, NGOs and consultants, covering equities and credit (2ii 2018). 
	%			
	%			In all this, it is important to emphasize that asset prices - based on market participants’ assumptions about changes in the yield-risk profile of securities - do not necessarily reflect the economic risks faced by a company. Thus, the price of assets , and the risk that their valuation will decrease, does not automatically reflect the underlying risks to which the companies are exposed. On the other hand, it should be noted that the return potential is optimized when the allocation of capital is as efficient as possible. If the capital is not allocated efficiently, the absolute benefit is also reduced. Signals issued by the financial markets in the form of portfolio reallocation choices or via shareholder engagement can thus help optimize the allocation of capital in the real economy, and help maximize long-term returns.
	%		\end{multicols}
	%	
	%		
	%		\newpage	
	\section*{} %NOTES AND DISCLAIMER
	\HeaderSingle{NOTES AND DISCLAIMER}
	
	\begin{multicols}{2}
		
		\textbf{Published Research}
		
		The methodology behind this scenario analysis, the accounting rules applied, and further information to the scenarios and data can be found in the following published research papers. 
		
		Accounting Principles: http://www.mdpi.com/2071-1050/ 10/2/328 
		
		Scenario Work: http://et-risk.eu/toolbox/ scenarios/ 
		
		Asset Level Data Analysis: http://2degrees-investing.org/ IMG/pdf/assetdata\_v0.pdf
		
		\textbf{Sources for the data and scenario analysis}
		
		Automobile data are from July 2017 and is provided by WardsAuto / AutoForecastSolutions. Power data is from July 2017 and is provided by GlobalData. Oil, gas and coal production data is from July 2017 and is provided by GlobalData. When linking asset data with companies, the data is used by the data providers mentioned above and, where possible, enriched with company data from Bloomberg. All financial data, as well as identification numbers for linking company data with financial instruments, come from Bloomberg. 
		
		
		\textbf{Sources}
		
		IPCC (2018) https://www.ipcc.ch/report/ar5/
		
		FSB (2018) https://www.fsb-tcfd.org/publications/final-recommendations-report/
		
		Aviva / ECIU (2015) https://www.aviva.com/media/thought-leadership/climate-change-value-risk-investment-and-avivas-strategicresponse/
		
		FSB (2018) https://www.fsb-tcfd.org/publications/final-recommendations-report/
		
		WoodMackenzie (2018) https://www.woodmac.com/news/ editorial/carbon-intensity-not-all-assets-are-created-equal/ 
		
		\textbf{Requesting a report}
		
		Insurers operating in California may request an individual report by emailing the Department of Insurance at \newline Francisco.Raygoza@insurance.ca.gov.
		
		\textbf{Disclaimer}
		
		The 2° Investing Initiative’s research is provided free of charge and 2Dii does not seek any direct or indirect financial compensation for its research. 2Dii is not an investment adviser and makes no representation regarding the advisability of investing in any particular company or investment fund or other vehicle. A decision to invest in any such investment fund or other entity should not be made in reliance on any of the statements set forth on this website and the analysis results. The information and analysis contained in this research report does not constitute an offer to sell securities or the solicitation of an offer to buy, or recommendation for investment, in any securities within the United States or any other jurisdiction. The information is not intended as financial advice. The research report and website results provide general information only. The information and opinions constitute a judgment as at the date indicated and are subject to change without notice. No representation or warranty, express or implied, is made by 2Dii as to their accuracy, completeness or correctness. 2Dii does not warrant that the information is up to date, nor does it take liability for errors in third-party sourced data.
	\end{multicols}
	
	\newpage
	
	
\end{document} 

